\documentclass{article}
\usepackage{ae,lmodern}
\usepackage[french]{babel}
\usepackage[utf8]{inputenc}
\usepackage[T1]{fontenc}

\usepackage{caption}
\captionsetup[figure]{labelformat=empty}

\PassOptionsToPackage{usenames,dvipsnames}{xcolor}
\usepackage{xcolor,colortbl}
\definecolor{DarkGrey}{HTML}{222222}
\definecolor{DarkBlue}{HTML}{004BA9}
\definecolor{DarkRed}{HTML}{CC1111}
\definecolor{DarkGreen}{HTML}{117711}
\definecolor{DarkOrange}{HTML}{CC7000}
\definecolor{LightGrey}{HTML}{DDDDDD}
\definecolor{LightBlue}{HTML}{F0F8FF}
\definecolor{codegreen}{rgb}{0,0.6,0}
\definecolor{codepurple}{rgb}{0.58,0,0.82}

\usepackage[cache=false]{minted}
\setminted[bash]{
   bgcolor=LightBlue,
   breaklines, breakanywhere,
   frame=single,
   autogobble
}
\usemintedstyle[python]{native}
\setminted[python]{
   bgcolor=black,
   breaklines, breakanywhere,
   autogobble
}

\usepackage{listings}
\usepackage{lstautogobble}
\lstdefinestyle{bash}{
    backgroundcolor=\color{DarkGrey},   
    commentstyle=\color{codegreen},
    keywordstyle=\color{magenta},
    numberstyle=\tiny\color{DarkGrey},
    stringstyle=\color{codepurple},
    basicstyle=\ttfamily\tiny\color{LightGrey},
    escapeinside={\%*}{*)},
    breakatwhitespace=false,         
    breaklines=true,                 
    captionpos=b,                    
    keepspaces=true,                 
    numbers=left,                    
    numbersep=5pt,                  
    showspaces=false,                
    showstringspaces=false,
    showtabs=false,
    showlines=false,
    tabsize=2
}

\usepackage{tikz}
\usetikzlibrary{calc,decorations.pathreplacing,arrows,arrows.meta,shapes,patterns, positioning}
\newcommand\BigLength{14.6em}
\newcommand\Height{2em}
\newcommand\Sep{0.6em}
\newcommand\Center{\BigLength*1/2}
\newcommand\BigBox{\BigLength+\Sep}
\newcommand\HalfBox{\BigLength*1/2-\Sep*1/4}
\newcommand\HalfLength{\BigLength*1/2-\Sep*5/4}
\newcommand\CenterL{\BigLength*1/4-\Sep*1/8}
\newcommand\CenterR{\BigLength*3/4+\Sep*1/8}
\tikzstyle{layer}=[rectangle,thick,text centered,
                     minimum height=\Height,minimum width=\BigLength]
\tikzstyle{short}=[rectangle,thick,text centered,
                     minimum height=\Height,minimum width=\HalfLength]
\tikzstyle{dibox}=[rectangle,thick,semitransparent,
                     minimum height=(\Height+\Sep)*2,minimum width=\BigBox]
\tikzstyle{vmbox}=[rectangle,thick,semitransparent,
                     minimum height=(\Height+\Sep)*3,minimum width=\HalfBox]
\tikzstyle{ctbox}=[rectangle,thick,semitransparent,
                     minimum height=(\Height+\Sep)*2,minimum width=\HalfBox]
\tikzstyle{vebox}=[rectangle,thick,semitransparent,
                     minimum height=(\Height+\Sep)*1,minimum width=\HalfBox]

\usepackage{hyperref}
\usepackage{grffile}

\renewcommand{\thesection}{TD\arabic{section}/5:}
\renewcommand{\thesubsection}{Exercise \arabic{subsection}:}


\setcounter{section}{0}

\begin{document}
%----------------------------------------------------------------------------------------
%========================================================================================
\section{Project}
%========================================================================================
%----------------------------------------------------------------------------------------

The goal of this project is to predict \textit{part\_assiette\_chomage\_partiel} one month ahead.
You will be guided through the differents steps that are fundamental in a Data Science project.
You still have an important part of autonomy in your code, methods, analysis and decisions.
Several interpretations often coexist as long as they are coherent. Do your best, 
be courageous and deploy your Data Science artillery!

%****************************************************************************************
\subsection{Set up}
%****************************************************************************************

\begin{enumerate}
    \item Load data:\\
    \textit{masse-salariale-et-assiette-chomage-partiel-mensuelles-du-secteur-prive\_modif.csv}\footnote{\href{https://www.data.gouv.fr/fr/datasets/masse-salariale-et-assiette-chomage-partiel-mensuelles-du-secteur-prive/}{Data is a modified version from this source}}
    \item See number of samples (rows) and features (columns)
    \item See data type
    \item Set \textit{dernier\_jour\_du\_mois} as index
    \item Cast index as datetime
    \item Sort index in ascending order
\end{enumerate}


%****************************************************************************************
\subsection{Data Analysis}
%****************************************************************************************

\begin{enumerate}
    \item Produce standard descriptive statistics
    \item Visualize data (many plots can be done in one line of code using Pandas and Seaborn)
\end{enumerate}

%****************************************************************************************
\subsection{Data Cleaning}
%****************************************************************************************

\begin{enumerate}
    \item Check for missing values (some are more subtle than a explicit NaN)
    \item Impute these missing values with at least 2 methods seen in the lectures, 
    don't delete them in this project (we need all the dates in a Times Series problem)
    \item Check and treat outlier(s)
\end{enumerate}

%****************************************************************************************
\subsection{Feature Engineering}
%****************************************************************************************

\begin{enumerate}
    \item Add a feature \textit{is\_year\_end}
    \begin{itemize}
        \item 1 when month is november or december
        \item 0 otherwise
    \end{itemize}
\end{enumerate}

%****************************************************************************************
\subsection{Prediction}
%****************************************************************************************

\begin{enumerate}
    \item Split your data into a train set (60\% of data) and a test set (40\%)
    \item Use a linear regression to predcit \textit{part\_assiette\_chomage\_partiel} 1 month ahead
    \begin{itemize}
        \item you should shift your features (in time) compared to your target
        \item find tutorials, there are a lot of them, its the only way toward autonomous learning!
    \end{itemize}
    \item How good is your prediction?
    \begin{itemize}
        \item Plot the predicted values on the same graph than the actual values
        \item Use metric(s) to evaluate your model on both the train and test sets
        \item Interpret the results
        \item Give advices to your (hypothetical) colleague to continue your work
    \end{itemize}
\end{enumerate}



%----------------------------------------------------------------------------------------
\subsubsection{Bonus}
%----------------------------------------------------------------------------------------

\begin{enumerate}
    \item Make a prediction without the added variable \textit{is\_year\_end}, what is the impact?
    \item Use a Ridge regression in place of the Linear regression (you might become happy about the results!)
    \item Use a \textbf{polynomial} regression to predcit 1 month ahead (find tutorials, there are a lot of them, and its the only way to learn autonomously!)
    \item Predict 2 months ahead, then 3 and 4 months ahead. If your code is written correctly, it should only require to manually change the value of a constant.
\end{enumerate}


\end{document}